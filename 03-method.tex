\section{\translate{Method}}\label{sec:method}
With regards to C- and AV-level diploma work, it is insufficient to merely
perform a practical construction or programming project. A systematic study must
also be carried out, e.g.\ an evaluation and analysis of the design or program.
The study should result in objective facts, preferably in the form of tables and
diagrams, into which your own conclusions are built in. The study can be a
verification of a design that meets the requirement specification, or a
comparison of competing alternatives. It is acceptable to allow users to answer
a questionnaire or be interviewed. It is also possible to evaluate web-pages and
other user interfaces according to usability criteria.

The method section is the point at which your chosen method and intended
procedure during the research are discussed. This section shall not be a
chronological diary filled with irrelevant details, but should contain
information given in such a way that it is understandable for the reader and
enables him/her to interpret your results and repeat your work, i.e.\ in order to
check the results. Here, the tools, assumptions, mathematical models,
performance measures and assessment criteria are presented. It is also at this
point that the means adopted for the evaluation and verification of the computer
programs and technical solution proposals are presented. This can include a test
plan to check that the structure works and criteria to assess its usefulness. In
research reports regarding natural science and technology this chapter is often
called ``Model'', ``System Model'' or ``Simulation Model''.

Justify your choice of methodology/model. This choice is very important, because
it could be the actual key to the result of your research. Comment on the
method's possible weaknesses and problems that may arise during actual
implementation. Refer to the problem wording in the introduction chapter. It is
possible, for example, to write ``problem P1 is attempted through the method M1
and problem P2 through\dots''

In your report, you should – depending on what the report is about– find
information about what you have investigated and how you have gathered and
processed data. Possible questionnaires, interview questions and the likes can
be presented as appendices. Detailed descriptions concerning experimental
formats of possible interest to those wanting to repeat the experiment should
also be included in this chapter.
