\section{\translate{Methodology}}\label{sec:method}
Summarize and introduce the methodology chapter. The method section is the point at which your
chosen method and intended procedure are presented. This section should contain information for the
reader to interpret your results and repeat your work, i.e.\ in order to check the results. Here, the
tools, assumptions, mathematical models, performance measures and assessment criteria are presented.

\subsection{\translate{Scientific method description}}\label{subsec:scientificmethod}
Explain your scientific method. Will you be using a quantitative or qualitative method? Most works
in computer engineer will be \emph{quantitative} studies. Will you follow the design science
methodology or some other methodology, such as action research method, descriptive research design,
correlational research design, or experimental research design.  Explain how you will scientifically
attack and work with each of the scientific goals/research questions. One paragraph per
goal/question is often enough.

\subsection{\translate{Project method description}/\translate{Work method
    description}}\label{subsec:projectmethod}
Explain your project and/or working method. Split the work into a number of achievable
milestones/phases.

Many theses work in computer engineering typically follow either a \emph{waterfall} inspired method
akin to the structure of this report: \emph{Theory} (in which you will gain more knowledge about the
requirements to be able to solve the problem),\emph{Pre-study} (where you will further refine
analyze your requirements, approach, design, etc.), \emph{Implementation} (where you will build some
artifact), \emph{Measure} (in which you will perform measurements on your artifact), and
\emph{Evaluate} (where you will analyze and evaluate your results and measurements). 

Or a more \emph{design science} inspired structure: \emph{Identify problems} (in which you define
the problem and show importance), \emph{Define objectives} (in which you define requirements, etc.),
\emph{Design & development} (where you will build some artifact), \emph{Demonstration} (where you
use the artifact to solve the problem), \emph{Evaluation} (where you observe its effectiveness),
\emph{Communication} (making your results public). 

Or a more \emph{design thinking} inspired structure: \emph{Empathize} (where you gain real insight
into users and their needs), \emph{Define} (where you define the problem statement in a
human-centered manner.), \emph{Ideate} (where you identify innovative solutions to the problem
statement you’ve created), \emph{Prototype} (where you create possible solutions), \emph{Test}
(where you test solutions to derive a deep understanding of the product and its users). 

Explain how you will work and do within each of these milestones/phases. Justify your choice of
methodology/model. What metrics will you use to evaluate your work? Comment on the method's possible
weaknesses and problems that may arise during actual implementation. A flowchart can help in
visualizing the phases/milestones. One paragraph per milestone/phase is often enough to write.


\subsection{\translate{Project evaluation method}}\label{subsec:evalmethod}
How will you look back upon your whole thesis work in the end and evaluate if it was a success?
Focus on your whole thesis project process and remember that this is different from your
measurements/evaluation of your implementation. How will you work with the questions in the
discussion chapter?
