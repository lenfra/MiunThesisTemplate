\section{\translate{Discussion} / \translate{Conclusion}}\label{sec:conclusion}
The conclusion/discussion (choose a heading) is a separate chapter in which the
results are analysed and critically assessed. At this point your own
conclusions, your subjective view, and explanations of the results are
presented.

If this chapter is extensive it can be divided up into more chapters or
sub-chapters i.e.\ one analysis or discussion chapter with explanations of and
critical assessment of the results, a concluding chapter where the most
important results and well supported conclusions are discussed and to sum it up
a chapter with suggestions for further research in the same area.

In this chapter it is of vital importance that a connection back to the aim of
the survey is made and thus the purpose is pointed out in a summary and analysis
of the results.  In this chapter you should also include answers to the
following questions: What is the project's news value and its most vital
contribution to the research or technology development? Have the project’s goals
been achieved? Has the task been accomplished? What is the answer to the opening
problem formula? Was the result as expected? Are the conclusions general, or do
they only apply during certain conditions? Discuss the importance of the choice
of method and model for the results. Have new questions arisen due to the
result?

The last question invites the possibility to offer proposals to others relevant
research, i.e.\ proposal points for measures and recommendations, points for
continued research or development for those wishing to build upon your work.

In technical reports on behalf of companies, the recommended solution to a
problem is presented at this stage and it is possible to offer a consequence
analysis of the solution from both a technical and layman perspective, for
example regarding environment, economy and changed work procedures. The chapter
then contains recommended measures and proposals for further development or
research, and thus to function as a basis for decision-making for the employer
or client.

