\section{\translate{Pre study}/\translate{Approach}}\label{sec:prestudy} 

Choose only one of the headlines. This chapter will be substantially different depending on your
thesis topic, direction, and scientific level. But in this chapter, you will present the different
options you have faced and made your choice between, or you will explain the pre-study that has led
up to your design, implementation, and your analysis of different requirement.        

\subsection{\translate{Requirements capturing}/\translate{Approach alternatives}}\label{subsec:solutionalt}
Explain the requirement capturing and the identified requirements, or the different approach alternatives.

\subsubsection{Alternative X/Approach X}\label{subsubsec:altx}
Present and summarize the potential approaches or requirements that you have considered. Make them
even in length and style, to be of equal in importance. Continue the list with all
alternatives/requirement, one heading for each.

\subsubsection{Another requirement/alternative}
Another requirement or potential approach.

\subsubsection{Etc.\ until all has been covered}
Another, etc.

\subsection{\translate{Requirement analysis}/\translate{Comparison of
    approaches}}\label{subsec:compareapproach}
Analyze the identified requirements, compare different approaches to another, benefits, drawbacks?
Pros vs cons? A table for clear comparison can be used, for example a Pugh matrix.

\subsection{\translage{Proposed approach}/\translate{Chosen approach}}\label{subsec:chosenapproach}
Present your proposed or chosen approach. It is very important to motivate why this was chosen over
the others. Connect back to your overall aim and problem statement, to ensure that the chosen
approach actually can answer your scientific goals/research questions.

