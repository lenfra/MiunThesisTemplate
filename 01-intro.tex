\section{\translate{Introduction}}\label{sec:intro}
During your previous education, you have probably come across relatively well
defined problem types as formulated by teachers, textbooks and teaching aids.
During project courses and exam work you are required to do a great deal of the
thinking by yourself in order to define and clarify the direction of the
assignment. This analysis should be presented in the report's introductory
chapter. By describing the problem or problem area chosen for study and the
reasons behind this choice, it should then be possible to write a general
introduction to the report. The introductory chapter relates to the content in
the project plan that will be presented some weeks after the diploma work has
started. The project plan should also contain a time plan for the work. The
project plan can also mention some of the intended sources to be read and
subsequently referred to in chapter 2, and also to contain some thoughts about
the method (see chapter 3) chosen in order to approach the problem. The
introduction making up chapter 1, may also contain sub-headings underneath. Try
to get to the point as soon as possible. In order to retain the reader’s
interest information concerning your work must be given within the first few
sentences. People only requiring a quick insight into the work will often only
read the report's summary, introduction and conclusions, since these sections
are usually written without the inclusion of highly technical and mathematical
details.

\subsection{\translate{Background and problem motivation}}\label{subsec:background}
\noindent 
In this sub-chapter you should try to quickly engage the readers' interest in
the problem area you have chosen to examine. Demonstrate that you are not only
familiar with any minor technical problems, but also have an understanding of
the context in which your problem emerges, that you can also describe it from a
non-technical perspective, and that you are aware of the practical benefits of
the technology you are examining or have knowledge of areas that your study
relates to. It is common that the first sentence contains an insightful
formulation or historical retrospective. Obviously it is not possible to be
absolutely certain with regards to the future, but you should express your
hypothesis in a balanced and objective manner in order to appear credible.

Examples: ``Humankind during historical times has\dots. The use of internet and
cellular telephony has grown since\dots. The next stage in the development is
expected to become\dots. This can lead to problems with… This study investigates if
the problem can be solved with the aid of\dots. This technology can become
especially interesting if in some years many more people\dots, and there is a
growing demand on the market after\dots''.
A technical report that is carried out on behalf of a company could start with:
“Within the organization there is an increased need for\dots and at the same time

\subsection{\translate{Overall aim}}\label{subsec:aim}
\noindent
The project's aim is an insightful description of the direction in which you
want to work, your hopes with regards to the possible outcomes of the project,
and of the projects' purpose. The hypothesis does not need to be clearly defined
or concrete. It can be an objective which may or may not be resolved or achieved
with any degree of certainty. It can be a problem formula of a high level, which
cannot be answered by the study's diagrams, tables and other objective results,
but which can be discussed in the report's concluding chapter. Examples: ``the
project's overall aim is to gain new knowledge within the organization
about\dots''.
``The project's aim is to identify the general valid principles for the
connection between parameter X and Y for everybody\dots''. ``The project's aim is to
find new technical solutions to problems in the following area:\dots.'' ``The
project's aim is to compare technology A with technology B as a solution to the
needs of C.'' ``The project aims to present a decision-making basis for\dots''
``The project aims to investigate whether or not it is realistic to expect that
technology A could be used for purpose B in the future.''

\subsection{\translate{Problem statement}}\label{subsec:problem}
Make a clear problem statement. Preferably in the form: The investigated problem
statement of this thesis is\dots 

The specified area must be answered later in the report's results, and in its
conclusion. The problem statement should be so clearly defined, that deciding
whether or not the problem has been resolved should be easy and understandable.

\subsection{\translate{Scientific goals}/
            \translate{Knowledge goals}/
            \translate{Research questions}}\label{subsec:researchquestion}

Split this hard problem into a number of achievable, concrete, and verifiable
scientific/knowledge goals or a number of research questions. Explain what the
scientific knowledge gained from this work will be.
\begin{enumerate}
  \item ???
  \item ???
  \item ???
\end{enumerate}

\subsection{\translate{Scope}}\label{subsec:scope}
\noindent
Explain what you have focused your work on and what you have chosen to not focus
on. What will be in focus? What will not be in focus?

\subsection{\translate{Outline}}\label{subsec:outline}
\noindent
Briefly describe the report's outline. ``Chapter 2 describes\dots''

\subsection{\translate{Division of work}}\label{subsec:contributions}
\noindent 
Describe which parts of the work that you have conducted yourself, and which
parts that you had help with i.e.\ carried out by colleagues. If the work is
carried out in a group the report should then explain how the tasks were divided
between authors. All co-authors should be credited in the work as a whole.
