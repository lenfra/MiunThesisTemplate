\section{\translate{Introduction}}\label{sec:intro} 
A brief introduction of your work and this chapter. Shortly explain the overall context surrounding
your work. Explain if it is a project proposed by a company or from a research group.

\subsection{\translate{Background and motivation}}\label{subsec:background}
\noindent 
Bakgrund och motivation) Explain the background and the motivation for this work. In this
sub-chapter you should try to quickly engage the readers' interest in the problem area you have
chosen to examine. Demonstrate that you are familiar with the technical problem, and that you
understand the context in which your problem emerges, that you can also describe it from a
non-technical perspective, and that you are aware of the practical benefits of the technology you
are examining. It is common that the first sentence contains an insightful formulation or historical
retrospective.  

\subsection{\translate{Overall aim and problem statement}}\label{subsec:aim}
\noindent
Explain why we are doing this, the higher reason for this work i.e.\ the purpose. The project's aim
should be written as an insightful description of the direction in which you want to work, your
hopes with regards to the possible outcomes of the project, and of the projects' higher purpose. Be
visionary and futuristic regarding the end goal of the thesis.

Make a clear problem statement.  Preferably in the form: The investigated problem of this thesis is…
The specified area must be answered later in the report's results and in its conclusion. The problem
statement should be so clearly defined, that deciding whether or not the problem has been resolved
should be easy and understandable.

\subsection{\translate{Scientific goals}/
            \translate{Research questions}}\label{subsec:researchquestion}

Choose only one of the headlines. Split this hard problem into a number of achievable, concrete,
measurable, and verifiable scientific goals or a number of research questions that you will answer.
Use adjectives and quantify them when possible. Explain what the scientific knowledge gained from
this work will be.

\begin{enumerate}
  \item ???
  \item ???
  \item ???
\end{enumerate}

\subsection{\translate{Scope}}\label{subsec:scope}
\noindent
Explain what you have focused your work on and what you have chosen to not focus
on. What will be in focus? What will not be in focus?

\subsection{\translate{Outline}}\label{subsec:outline}
\noindent
Briefly describe the report's outline. ``Chapter 2 describes\dots''

\subsection{\translate{Division of work}}\label{subsec:contributions}
\noindent 
Describe which parts of the work that you have conducted yourself, and which
parts that you had help with i.e.\ carried out by colleagues. If the work is
carried out in a group the report should then explain how the tasks were divided
between authors. All co-authors should be credited in the work as a whole.
