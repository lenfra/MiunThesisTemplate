\section{\translate{Construction}}\label{sec:construction}
The Design or Construction chapter often appears in technical reports, but not
always in scientific reports. Here, the analysis of the problem is implemented
and a technical requirement specification is formulated. At this stage, the most
important principles in the suggested alternatives for solution are described
and formulated in preparation for evaluation at a later point in the report. The
description is sometimes placed before, but generally after the
methodology/model chapter, if included at all.

The reader is seldom interested in extremely detailed documentation of computer
program code, algorithms, electrical circuit diagrams, user guidance, etc. Such
details are placed in the appendices.

As mentioned in the Introduction chapter you have during earlier studies mainly
worked with small well defined tasks that have taken minutes or as most hours to
solve. In comparison an exam work or a project course can sometimes appear to be
an almost overwhelming amount of information because it is so extensive, and
this may cause anxiety with regards to where to start. One way to facilitate big
projects is to use the top-down-method, i.e.\ to divide the problem or the
structure into smaller problem parts or system parts, and to state specification
of  requirements, problem analysis and proposed solution for each part.
Eventually small and concrete information will have been identified with similar
characteristics to those found in your previous studies.

It is not always practically possible to apply the top-down-method, since the
problem may be too complex and initially very difficult to visualise the
complete overview. It might prove necessary to alternate between the top-down 
and bottom-up-method. The latter means that you start with parts already known
to you and from simple problems that have been tackled previously you  make use
of that knowledge for aspects that you expect to resolve at a later stage in the
project. Gradually increase these parts into the bigger systems and problems and
then pursue the direction of project's objective.

The top-down-method has the advantage of giving the report a solid structure,
which makes it easier for the reader. The documentation therefore often follows
the top-down-method. It is thus possible to divide the structure part into
several chapters, and to name them after each problem part and system part, i.e.
``Specification of requirements'', ``Algorithms'', ``User interface'', ``Program
documentation'', ``Prototype'' and ``Implementation''.
